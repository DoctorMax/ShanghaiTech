\documentclass[english,onecolumn,UTF8]{IEEEtran}
\usepackage[T1]{fontenc}
\usepackage[latin9]{luainputenc}
\usepackage[letterpaper]{geometry}
\geometry{verbose}
\usepackage{amsfonts}
\usepackage{amsmath}
\usepackage{babel}
\providecommand{\U}[1]{\protect\rule{.1in}{.1in}}
\topmargin            -18.0mm
\textheight           226.0mm
\oddsidemargin      -4.0mm
\textwidth            166.0mm
\def\baselinestretch{1.5}



\begin{document}

\begin{center}
\textbf{SI231 - Matrix Computations, Fall 2020-21} \\ Homework Set \#1\\
\texttt{\textbf{Name:}}   	\texttt{ Sujinning }  		\hspace{1bp}
\texttt{\textbf{Major:}}  	\texttt{ Master in IE } 	\\
\texttt{\textbf{Student No.:}} 	\texttt{ 2020232125}     \hspace{1bp}
\texttt{\textbf{E-mail:}} 	\texttt{ sujn@shanghaitech.edu.cn}
\par\end{center}

\section{Orthogonality}
\textbf{Problem1}
\begin{enumerate}
	 
	\item \textbf{Solution:}
	
	 Firstly for any vector \(x \in \mathcal{N} \)(A), \quad we have  \(Ax=0\), which means for every row in A , the dot product of the row and $x$ is 0. Obviously	, vectors \(x\) \( \in \mathcal{N} \) (A) and linear conbinations of vectors in \(\mathcal{R}\)(A\(^{T}\)) are orthogonal \\
	From the hint $dim(\mathcal{N}(A))+ dim(\mathcal{R}(A^{T}))=n$ \\
	We have the union of their basis are still linear independent and they can span a $\mathbb{R}$ space \\
	Then $\mathbb{R} = \mathcal{N}(A) \oplus \mathcal{R}(A^{T})$ is proved
	

	\item \textbf{Solution:}

	let \(rank(A)=p,rank(B)=q\),\ then we have a set of basis \([\ \alpha_{1},...\alpha_{p},\beta_{p+1},\beta_{p+2}...\beta_{p+q}\ ]\) \\
	Then we have \(A+B\) is the linear combination of the basis from the set \\
	Resultly \(rank(A+B) = dim(A+B) \leq p+q =rank(A)+rank(B)\)\\
	proved
	




	
	\item \textbf{Solution:}
	\begin{enumerate}
	\item We know \(AB\) can be regarded as the linear combination of matrix A 's columns. So \(rank(AB) \leq rank(A)\) 
	the same, \(AB\) can also be regarded as the linear combination of matrix B 's rows. So \(rank(AB) \leq  rank(B^{T})=rank(B)\)\\
	
	\item We know \(AB\) can be regarded as the linear combination of matrix A 's columns. So \(rank(AB) \leq rank(A)\) 
	and \(AB\) can also be regarded as the linear combination of matrix B 's rows.\\
	Assume A has no full-column rank,then we know the linear combination of A 's column should have a dimension lower than n, which means $rank(AB)$ can not be n. The same, if B has no full-row rank, $rank(AB)$ can not have rank n.
	Therefore,only when A has full-column rank and B has full-row rank can we have \(rank(AB)=n\)
	
	\end{enumerate}

	\item \textbf{Solution:}\\
	$\mathcal{R}(A|B)$ can be written as the linear combination of columns of matrix $A$ and $B$, which means vector $x$ in $\mathcal{R}(A|B)$  can be written as the linear combination of $[\alpha_{1},\alpha_{2}......\alpha_{n},\beta_{1},......\beta_{p}]$. Similarly,  vectors $a$ in $A$ can be writen as the linear combination $[\alpha_{1}......\alpha_{n}]$ and vectors $b$ in $B$ can be writen as the linear combination $[\beta_{1}......\beta_{p}]$, which means $x$ can be writen as the linear combination of $a$ and	$b$. Then x are in $\mathcal{R}(A)+\mathcal{R}(B)$. which means $\mathcal{R}(A|B) \subseteq (\mathcal{R}(A)+\mathcal{R}(B))$
	\\ 
	Obversiouly, vectors in $\mathcal{R}(A)$ and $\mathcal{R}(B)$ can be writen as the linear combination of $[\alpha_{1},\alpha_{2}......\alpha_{n},\beta_{1},......\beta_{p}]$, which means $(\mathcal{R}(A)+\mathcal{R}(B)) \subseteq \mathcal{R}(A|B)$
	\\
	Finally, from $\mathcal{R}(A|B) \subseteq (\mathcal{R}(A)+\mathcal{R}(B))$ and $(\mathcal{R}(A)+\mathcal{R}(B)) \subseteq \mathcal{R}(A|B)$, $\mathcal{R}(A|B)=\mathcal{R}(A)+\mathcal{R}(B)$ is proved 

	\item \textbf{Solution:}

	 We have \(dim(\mathcal{R}(A)+\mathcal{R}(B))=dim(\mathcal{R}(A))+dim(\mathcal{R}(B))-dim(\mathcal{R}(A)\cap\mathcal{R}(B)))\)\\
	From\ (4),\ we have\ \(dim(\mathcal{R}(A)+\mathcal{R}(B))=dim(\mathcal{R}(A|B))\)\\
	Then, $rank(A|B)=dim(\mathcal{R}(A|B))=dim(\mathcal{R}(A))+dim(\mathcal{R}(B))-dim(\mathcal{R}(A)\cap\mathcal{R}(B))\\
	=rank(A)+rank(B)-dim(\mathcal{R}(A)\cap\mathcal{R}(B))$
	



\end{enumerate}
	
\section{UNDERSTAND SPAN,SUBSPACE}
	\textbf{Problem1}

\begin{enumerate}
	
	\item \textbf{Solution:}
	\begin{enumerate}
	\item For $span(\mathcal{S}) \subseteq \mathcal{M}$\\
	for vector $x$ in $span(\mathcal{S})$, it can be writen as the linear combination of $[\ v_{1}...v_{n} \ ]$.\\ Consider $\mathcal{M}=\cap_{s\subseteq\mathcal{V}}\mathcal{V}$ , we have $S \subseteq \mathcal{M} $.Resultly,all the vectors in $span(\mathcal{S})$ can be writen as the linear combination of $[\ v_{1}...v_{n} \ ]$,which are in $S$ \ , belonging to $\mathcal{M}$
	\item For $\mathcal{M}\subseteq span(\mathcal{S})$ \\
	Obviously, $span(\mathcal{S})$ is one of the subspace that contain vectors set $\mathcal{S}$, so $span(\mathcal{S})$ is a special $\mathcal{V}$. It is proved that $\mathcal{M}\subseteq span(\mathcal{S})$ because $\mathcal{M}=\cap_{s\subseteq\mathcal{V}}\mathcal{V}$
	
	\end{enumerate}

\end{enumerate}  
~\\

\section{BASIS,DIMENSION AND PROJECTION}

\textbf{problem1}
\begin{enumerate}
	\item\textbf{Solution:} \\
	The dimension is \(n+1\)
	\item\textbf{Solution:} \\
	 	The dimension is \(\frac{n(n+1)}{2}\)
\end{enumerate}

\textbf{problem2}
\begin{enumerate}
	\item \textbf{Solution:}\\
	\begin{enumerate}
		\item 
		Assume $R=
		\begin{bmatrix}
		a & b \\
		c & d \\
		\end{bmatrix}
		$,\ 
		From $RR^{T}=I$ and det$(R)=1$ we have:
		\begin{center}
		$ a^{2}+b^{2}=1$\\
		$ac-bd=0$\\
		$c^{2}+d^{2}=1$\\
		$ad-bc=1$\\			
		\end{center}
	After solve the equation we have\\
		\begin{center}
		$R=\begin{bmatrix}
		sin(\theta) & cos(\theta)\\
		-cos(\theta) & sin(\theta)\\
		\end{bmatrix}
		,\theta \in [0,2\pi)
		$		
		\end{center}
~\\	
\item
		
			$\textbf{R}x=\begin{bmatrix}
				cos(\frac{5\pi}{6})\\
				sin(\frac{5\pi}{6})\\
			 \end{bmatrix}
			$
	
	\end{enumerate}
	~\\
	\item\textbf{Solution}
	\begin{center}
		$QHx
		=(I-uu^{T})(I-2uu^{T})x$\\
		\qquad\qquad$=(I-3uu^{T}+2u(u^{T}u)^{T}u)x
		$\\
		$=(I-uu^{T})x=Qx$
		\end{center}
	Then
	\begin{align*}
	||Hx-QHx||_{2}
	&=||(I-Q)Hx||_{2}\\
	&=||uu^{T}(I-2uu)^{T}x||_{2}\\
	&=||uu^{T}x||_{2}\\
	&=||I-Qx||_{2}\\
	&=||x-Qx||_{2}
	\end{align*}
	Resultly,$Hx$ is a reflection of $x$ with respect to $\mathcal{H}_{u}$
	
\end{enumerate}

~\\

\section{DIRECT SUM}

\noindent\textbf{Problem1}\\
\textbf{Solution}\\
Assume that $\mathcal{V}$ is a n-dimension space, then $\mathcal{B}$ can be written as $[v_{1},v_{2}......v_{n}]$ in which vectors are linear independent.subsets $\mathcal{B}_{1}$ and $\mathcal{B}_{2}$ can be writen as $\mathcal{B}_{1}=[v_{1}......v_{p}]$ and $\mathcal{B}_{2}=[v_{p+1}......v_{n}]$ .Then we know vectors in $span(\mathcal{B}_{1})$ are linear combinations of vectors in $\mathcal{B}_{1}$, while vectors in $span(\mathcal{B}_{1})$ can not be written by linear combination of vectors in $\mathcal{B}_{2}$. So $span(\mathcal{B}_{1}) \cap span(\mathcal{B}_{2})= \emptyset$
Besides,dimension of $\mathcal{B}_{1}$ should be $p$ and dimension of $\mathcal{B}_{2}$ should be  $n-p$ which are the same as $span(\mathcal{B}_{1})$ and $span(\mathcal{B}_{2})$.
So $dim(span(\mathcal{B}_{1}))+dim(span(\mathcal{B}_{2}))=n$ and $span(\mathcal{B}_{1}) \cap span(\mathcal{B}_{2})= \emptyset$ , which prove $\mathcal{V}=span(\mathcal{B}_{1}) \oplus span(\mathcal{B}_{2})$

~\\
\textbf{Problem2}\\
\textbf{Solution}\\

we have d-dimension subspace $\mathcal{S}$
Assume $v=[v_{1},......v_{n}]$ is a set of basis of $\mathcal{V}$. Then there must be d vectors in the set and we can span the d vectors  to construct $\mathcal{S}$. Assume $[v_{1},......v_{d}]$ is the a set of basis for $\mathcal{S}$,then the rest vectors in $v=[v_{1},......v_{n}]$ are $v_{rest}=[v_{d+1},......v_{n}]$
Then from Problem1 we have 
$\mathcal{S} \cap span(v_{rest})=\emptyset $
and $dim(S)+dim(span(v_{rest}))=n$ 
Finally we have $\mathcal{V}=\mathcal{S} \oplus span(v_{rest})$,which means $\mathcal{V}=\mathcal{S} \oplus \mathcal{T}$ is proved
~\\
\section{UNDERSTANDING THE MATRIX NORM}

\textbf{Problem1}

\begin{enumerate}
\item\textbf{Solution}\\
The result of $Ax$ is the linear combinaton of column vectors $[\alpha_{1},......\alpha_{n}]$ in $A$ , if we add a 1-norm to the result,$||Ax||_{1}=||x_{1}a_{1}+.......+x_{n}a_{n}||_{1}\leq max(||a_{1}||_{1},...,||a_{n}||_{1})$, equality holds when\\ $||a_{i}||_{1}=max(||a_{1}||_{1},...,||a_{n}||_{1})$ and $x_{i}=1$ \\
so \\
$
\max_{||x||_{1}=1}||Ax||_{1} = \max(||a_{1}||_{1},...,||a_{n}||_{1})=\max\limits_{j}\sum\limits_{i}^{m}|a_{ij}|
$

\item\textbf{Solution}\\
like (1) , if we add a $\infty-norm$ to the result\\
\begin{align*}
	&||Ax||_\infty\\
	=&||x_1a_1+\cdots+x_na_n||\\
	=&||||a_1||_1+\cdots+||a_n||_1||_\infty\\
	=&\max_{1\leq i<m}\sum_{j=1}^n|a_ij|
\end{align*}
Equality holds when we firstly choose the a largest absolute sum row and secondly for every element in row vector $a_{kj},j=1,...,n, x_ja_{kj}=|a_{kj}|$ \\
so 
\begin{align*}
	||A||_{\infty} &=\max_{||x||_{\infty}=1}||Ax||_{\infty}\\
	&=\max_{i}\sum_{j}^{m}|a_{ij}|
\end{align*}
\end{enumerate}

~\\
\section{UNDERSTANDING THE HOLDER INEQUALITY}

\textbf{problem1}
\begin{enumerate}
\item \textbf{Solution}\\
$f^{'}(t)=\lambda - \lambda t^{\lambda-1}, 0<\lambda<1$. \quad
when $0<t<1,f^{'}(t)<0$ and when $t>1,f^{'}(t)>0$\quad
So $f_{min}(t)=f(0)=0$
\\
\\
Let $t=\frac{\alpha}{\beta}$,\quad then $f(\frac{\alpha}{\beta})=(1-\lambda)+\lambda(\frac{\alpha}{\beta})-(\frac{\alpha}{\beta})^{\lambda}\geq 0$, when we mutiple $\beta$ on both side , we get:\\
$(1-\lambda)\beta+\lambda\alpha-\alpha^{\lambda}\beta^{1-\lambda}\geq 0 \Rightarrow \alpha^{\lambda}\beta^{\lambda}\leq \lambda \alpha +(1-\lambda)\beta$ 
~\\

\item\textbf{Solution}\\
Let $\alpha = |\hat{x_{i}}|^{p}$ , $\beta = |\hat{y_{i}}|^{q}$, $\lambda=\frac{1}{p}$, then we have:\\
\begin{center}
$ |\hat{x_{i}}\hat{y_{i}}|\leq\frac{1}{p}|\hat
{x_{i}|^{p}}+\frac{1}{q}|\hat{y_{i}}|^{q}
$
\end{center}
So
\begin{center}
	$ \sum_{i=1}^n|\hat{x_{i}}\hat{y_{i}}|\leq \sum_{i=1}^n\frac{1}{p}|\hat
	{x_{i}|^{p}}+\sum_{i=1}^n\frac{1}{q}|\hat{y_{i}}|^{q}=\frac{1}{p}+\frac{1}{q}=1
	$
\end{center}
~\\
\item\textbf{Solution}\\
$\sum_{i=1}^n|\hat{x_{i}}\hat{y_{i}}|\leq 1
\Rightarrow \sum_{i=1}^n|\hat{x_{i}}\hat{y_{i}}|\leq||x||_{p}||y||_{q}\Rightarrow |x^{T}y|\leq ||x||_{p}||y||_{q}$\\
proved

\end{enumerate}

















\end{document}
