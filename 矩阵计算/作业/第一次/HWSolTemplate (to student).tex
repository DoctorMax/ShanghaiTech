\documentclass[english,onecolumn,UTF8]{IEEEtran}
\usepackage{CTEX}
\usepackage[T1]{fontenc}
\usepackage[latin9]{luainputenc}
\usepackage[letterpaper]{geometry}
\usepackage{amssymb}
\geometry{verbose}
\usepackage{amsfonts}
\usepackage{amsmath}
\usepackage{babel}
\providecommand{\U}[1]{\protect\rule{.1in}{.1in}}
\topmargin            -18.0mm
\textheight           226.0mm
\oddsidemargin      -4.0mm
\textwidth            166.0mm
\def\baselinestretch{1.5}



\begin{document}

\begin{center}
\textbf{SI231 - Matrix Computations, Fall 2020-21} \\ Homework Set \#1\\
\texttt{\textbf{Name:}}   	\texttt{ Liu Chang }  		\hspace{1bp}
\texttt{\textbf{Major:}}  	\texttt{ Master in IE } 	\\
\texttt{\textbf{Student No.:}} 	\texttt{ 2020232161}     \hspace{1bp}
\texttt{\textbf{E-mail:}} 	\texttt{ liuchang3@shanghaitech.edu.cn}
\par\end{center}

\section{UNDERSTANDING RANK, RANGE SPACE AND NULL SPACE}
	
	\textbf{Solution 1:}
	\begin{enumerate}
	\item 
		$\because$ $\mathcal{R}(\mathrm{A}^T)=\mathcal{N}(\mathrm{A})^\perp \; \mbox{and} $ 
		$\because$ $N(\mathrm{A})\oplus N(\mathrm{A})^\perp =\mathbb{R}^n$\\
		$\therefore$ $\mathcal{N}(\mathrm{A})\oplus \mathcal{R}(\mathrm{A}^T)=\mathbb{R}^n$

	\item 
		$\because$ $\mathcal{R}(\mathrm{A}+\mathrm{B})\in \mathcal{R}(\mathrm{A})+\mathcal{R}(\mathrm{B})$\\
		$\therefore$ $\mbox{rank}(\mathrm{A}+\mathrm{B)}=\mbox{dim}\mathcal{R}(\mathrm{A}+\mathrm{B})\leq \mbox{dim}\Big(\mathcal{R}(\mathrm{A})+\mathcal{R}(\mathrm{B})\Big)=\mbox{dim}\mathcal{R}(\mathrm{A})+\mbox{dim}\mathcal{R}(\mathrm{B})-\mbox{dim}\Big(\mathcal{R}(\mathrm{A})\cap \mathcal{R}(\mathrm{B})\Big)\leq \mbox{dim}\mathcal{R}(\mathrm{A})+\mbox{dim}\mathcal{R}(\mathrm{B})=\mbox{rank}(\mathrm{A})+\mbox{rank}(\mathrm{B})$

	\item 
		$\because$ $\mathrm{A}\mathrm{B}=(a_1,a_2,\cdots a_n)(b_1,b_2,\cdots b_n)^T$\\
		$\therefore$ $\mathrm{A}\mathrm{B}$ is the linear combination of  matrix A 's columns and matrix B 's rows.\\
		$\therefore$ $\mbox{rank}(\mathrm{A}\mathrm{B})\leq \mbox{rank}(\mathrm{A})$ and $\mbox{rank}(\mathrm{A}\mathrm{B})\leq \mbox{rank}(\mathrm{B}^T)=\mbox{rank}(\mathrm{B})$\\
		$if$ A has full-column rank and B has full-row rank\\
		$then$ $\mbox{rank}(\mathrm{A}\mathrm{B})=\mbox{rank}(\mathrm{A})=\mbox{rank}(\mathrm{B})=n$\\
		$if$ A has no full-column rank or B has no full-row rank\\
		$then$ $\mbox{rank}(\mathrm{A}\mathrm{B})\leq n$

	\item 
		$\mathcal{R}(A|B)$ means all the linear combination of  $(a_1,a_2,\cdots a_n,b_1,b_2,\cdots b_p)$\\
		$\mathcal{R}(A)+\mathcal{R}(B)$ means all the linear combination of  $(a_1,a_2,\cdots a_n)$ and $(b_1,b_2,\cdots b_p)$\\
		Every vectors $x$ in $\mathcal{R}(A|B)$ and in $\mathcal{R}(A)+\mathcal{R}(B)$ both can be written in the same form like $\sum^n_{i=1}x_ia_i+\sum^p_{i=1}y_ib_i$\\
		Finally,  $\mathcal{R}(A|B)=\mathcal{R}(A)+\mathcal{R}(B)$ is proved

	\item 
		$\because$ $\mbox{dim}(\mathcal{R}(A)+\mathcal{R}(B))=\mbox{dim}(\mathcal{R}(A))+\mbox{dim}(\mathcal{R}(B))-\mbox{dim}(\mathcal{R}(A)\cap\mathcal{R}(B)))$ and $\because$$\mbox{dim}(\mathcal{R}(A)+\mathcal{R}(B))=\mbox{dim}(\mathcal{R}(A|B))$\\
		$\therefore$ $\mbox{rank}(A|B)=\mbox{dim}(\mathcal{R}(A|B))=\mbox{dim}(\mathcal{R}(A))+\mbox{dim}(\mathcal{R}(B))-\mbox{dim}(\mathcal{R}(A)\cap\mathcal{R}(B))=\mbox{rank}(A)+\mbox{rank}(B)-\mbox{dim}(\mathcal{R}(A)\cap\mathcal{R}(B))$
	\end{enumerate}
	
\section{UNDERSTAND SPAN,SUBSPACE}
	\textbf{Solution 1:}
	\begin{enumerate}
	\item 
		To prove $span(\mathcal{S})\subseteq \mathcal{M}$\\
		$\because$ $\mathcal{M}=\cap_{s\subseteq\mathcal{V}}\mathcal{V}$\\
		$\therefore$ $S=\{v_1,\cdots v_n\} \subseteq \mathcal{M}$\\
		$\because$ $\mathcal{M}$ is a vactor space\\
		$\therefore$ $\sum^n_{i=1}x_iv_i\subseteq \mathcal{M}$\\
		$\therefore$ $span(\mathcal{S})\subseteq \mathcal{M}$
	\item
		To prove $\mathcal{M}\subseteq span(\mathcal{S})$\\
		$\because$ $span(\mathcal{S})$ is one of the subspace which contain $\mathcal{S}$\\
		$\therefore$ $\mathcal{S}$ is one of the $\mathcal{V}$ which contain $\mathcal{M}$\\
		$\therefore$ $\mathcal{M}\subseteq span(\mathcal{S})$

	\item 
		$\because$ $span(\mathcal{S})\subseteq \mathcal{M}$ and $\mathcal{M}\subseteq span(\mathcal{S})$\\
		$\therefore$ $\mathcal{M}=span(\mathcal{S})$
	\end{enumerate}

~\\

\section{BASIS,DIMENSION AND PROJECTION}

\textbf{Solution 1:}
\begin{enumerate}
	\item
		The dimension is $n+1$
	\item
	 	The dimension is $\frac{n(n+1)}{2}$
\end{enumerate}

\textbf{Solution 2:}
\begin{enumerate}
	\item \textbf{Rotations}\\
	\begin{enumerate}
		\item 
		Assume 
		$R=
		\begin{bmatrix}
		a & b \\
		c & d \\
		\end{bmatrix}
		$\\
		$\because$ $RR^{T}=I$ and det$(R)=1$\\
		\[ \left \{
		\begin{array}{l}
			a^{2}+b^{2}=1\\
			ac-bd=0\\
			c^{2}+d^{2}=1\\
			ad-bc=1
		\end{array}
		 \right.\]
		
	Therefore\\
		\begin{center}
		$R=\begin{bmatrix}
		sin(\theta) & cos(\theta)\\
		-cos(\theta) & sin(\theta)\\
		\end{bmatrix}
		,\theta \in [0,2\pi)
		$		
		\end{center}
~\\	
	\item
		$\textbf{R}x=(cos(\frac{5\pi}{6}),sin(\frac{5\pi}{6}))^T$
	\end{enumerate}
	~\\
	\item\textbf{Reflections}
	\begin{align*}
		QHx
		&=(I-uu^{T})(I-2uu^{T})x\\
		&=(I-3uu^{T}+2u(u^{T}u)^{T}u)x\\
		&=(I-uu^{T})x=Qx
	\end{align*}
	Therefore
	\begin{align*}
		||Hx-QHx||_{2}
		&=||(I-Q)Hx||_{2}\\
		&=||uu^{T}(I-2uu)^{T}x||_{2}\\
		&=||uu^{T}x||_{2}\\
		&=||I-Qx||_{2}\\
		&=||x-Qx||_{2}
	\end{align*}
	$\therefore$ $Hx$ is a reflection of $x$ with respect to $\mathcal{H}_{u}$
	
\end{enumerate}

~\\

\section{DIRECT SUM}

\textbf{Solution 1:}
	\begin{enumerate}
	\item
		According to the question, assume $dim(\mathcal{V})=n$, $\mathcal{B}=(v_1, v_2,\cdots v_n)$ in which vectors are linear independent, $\mathcal{B}_1=(v_1, v_2,\cdots v_m)$, $\mathcal{B}_2=(v_{m+1}, v_{m+2},\cdots v_n)$\\
		$\therefore$ $span(\mathcal{B}_{1})=\sum _{i=1}^mx_iv_i$ and $span(\mathcal{B}_{2})=\sum _{i=m+1}^ny_iv_i$\\
		$\therefore$ $span(\mathcal{B}_{1}) \cap span(\mathcal{B}_{2})= \emptyset$ and $dim(span(\mathcal{B}_{1}))+dim(span(\mathcal{B}_{2}))=m+(n-m)=n$\\
		$\therefore$ $\mathcal{V}=span(\mathcal{B}_{1}) \oplus span(\mathcal{B}_{2})$
	\end{enumerate}

\textbf{Solution 2:}
	\begin{enumerate}
	\item
		According to the question, assume $dim(\mathcal{V})=n$, the basis is $(v_1, v_2,\cdots v_n)$ in which vectors are linear independent, $\mathcal{S}=span(v_1, v_2,\cdots v_d)$, $\mathcal{T}=span(v_{rest})$\\
		$\therefore$ $\mathcal{S}\cap \mathcal{T}= \emptyset$ and $dim(\mathcal{S})+dim(\mathcal{T})=d+(n-d)=n$\\
		$\therefore$ $\mathcal{V}=\mathcal{S} \oplus \mathcal{T}$
	\end{enumerate}
~\\
\section{UNDERSTANDING THE MATRIX NORM}

\textbf{Solution 1:}

\begin{enumerate}
	\item
		$\because$ The result of $Ax$ is the linear combinaton of column vectors $[\alpha_{1},......\alpha_{n}]$ in $A$\\
		$\therefore$ $if$ we add a 1-norm to the result,$||Ax||_{1}=||x_{1}a_{1}+.......+x_{n}a_{n}||_{1}\leq max(||a_{1}||_{1},...,||a_{n}||_{1})$, equality holds when $||a_{i}||_{1}=max(||a_{1}||_{1},...,||a_{n}||_{1})$ and $x_{i}=1$ \\
		$\therefore$ $\max_{||x||_{1}=1}||Ax||_{1} = \max(||a_{1}||_{1},...,||a_{n}||_{1})=\max\limits_{j}\sum\limits_{i}^{m}|a_{ij}|$

	\item
		$if$ we add a $\infty-norm$ to the result
		\begin{align*}
			||Ax||_\infty
			=&||x_1a_1+\cdots+x_na_n||\\
			=&||||a_1||_1+\cdots+||a_n||_1||_\infty\\
			=&\max_{1\leq i<m}\sum_{j=1}^n|a_ij|
		\end{align*}
		Equality holds when we firstly choose the a largest absolute sum row and secondly for every element in row vector $a_{kj},j=1,...,n, x_ja_{kj}=|a_{kj}|$ \\
		$\therefore$ $||A||_{\infty}=\max_{||x||_{\infty}=1}||Ax||_{\infty}=\max_{i}\sum_{j}^{m}|a_{ij}|$
\end{enumerate}

~\\
\section{UNDERSTANDING THE HOLDER INEQUALITY}

\textbf{Solution 1:}
\begin{enumerate}
	\item
		$f^{'}(t)=\lambda - \lambda t^{\lambda-1}, 0<\lambda<1$.
		when $\left \{ 
		\begin{array}{l}
			0<t<1,f^{'}(t)<0\\
			t>1,f^{'}(t)>0
		\end{array}
		\right.$\\
		$\therefore$ $f_{min}(t)=f(0)=0$\\
		Let $t=\frac{\alpha}{\beta}$,\quad then $f(\frac{\alpha}{\beta})=(1-\lambda)+\lambda(\frac{\alpha}{\beta})-(\frac{\alpha}{\beta})^{\lambda}\geq 0$, when we mutiple $\beta$ on both side , we get:\\
		$(1-\lambda)\beta+\lambda\alpha-\alpha^{\lambda}\beta^{1-\lambda}\geq 0 \Rightarrow \alpha^{\lambda}\beta^{\lambda}\leq \lambda \alpha +(1-\lambda)\beta$ 
	\item
		Let $\alpha = |\hat{x_{i}}|^{p}$ , $\beta = |\hat{y_{i}}|^{q}$, $\lambda=\frac{1}{p}$\\
		$\therefore$ $ |\hat{x_{i}}\hat{y_{i}}|\leq\frac{1}{p}|\hat{x_{i}|^{p}}+\frac{1}{q}|\hat{y_{i}}|^{q}$\\
		$\therefore$ $ \sum_{i=1}^n|\hat{x_{i}}\hat{y_{i}}|\leq \sum_{i=1}^n\frac{1}{p}|\hat{x_{i}|^{p}}+\sum_{i=1}^n\frac{1}{q}|\hat{y_{i}}|^{q}=\frac{1}{p}+\frac{1}{q}=1$
	\item
		$\sum_{i=1}^n|\hat{x_{i}}\hat{y_{i}}|\leq 1\Rightarrow \sum_{i=1}^n|\hat{x_{i}}\hat{y_{i}}|\leq||x||_{p}||y||_{q}\Rightarrow |x^{T}y|\leq ||x||_{p}||y||_{q}$\\
		Proved.

\end{enumerate}







\end{document}
